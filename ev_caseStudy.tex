\documentclass[a4paper,12pt]{article}
\usepackage[a4paper,margin=2.5cm]{geometry}
\usepackage{graphicx}
\usepackage{setspace}
\usepackage{titlesec}

% Formatting section titles
\titleformat{\section}{\Large\bfseries}{\thesection}{1em}{}
\titleformat{\subsection}{\large\bfseries}{\thesubsection}{1em}{}
\geometry{margin=1in}
\setstretch{1.5}
\begin{document}

% ---------------- Title Page ----------------
\begin{titlepage}
    \centering
    \vspace*{0.5cm}
    {\Huge \textbf{Case Study Report on}}\\[0.6cm]
    {\LARGE \textbf{BYD YangWang u8 EV}}\\[1.5cm]

    \textbf{Submitted by}\\[0.3cm]
    {\large Prathamesh Prakash Shinde}\\[1.2cm]

    \textbf{Under Guidance of}\\[0.3cm]
    {\large Prof. A .S. Mulnai}\\[1.5cm]

    \includegraphics[width=3cm]{college-logo.png}\\[0.8cm]

    {\large Sant Dnyaneshwar Shikshan Sanstha’s}\\[0.3cm]
    {\Large \textbf{ANNANSAHEB DANGE COLLEGE OF ENGINEERING AND TECHNOLOGY, ASHTA}}\\[0.3cm]
    (An Empowered Autonomous Institute)\\[0.6cm]

    {\large Academic Year: 2025–26}

    \vfill
\end{titlepage}

% LaTeX content for EV Case Study on BYD Yangwang U8
% Paste this after your first page (do not include another \begin{document}).

\newpage
\begin{center}
    \Huge\textbf{Abstract}
\end{center}
\vspace{1cm}

\noindent
This case study focuses on the \textbf{BYD Yangwang U8}, a luxury electric SUV that represents cutting-edge electric vehicle (EV) technology. The report examines its technical specifications, design, environmental impact, and compares it with similar EVs in the market. It also includes geotagged images of the vehicle, highlighting its real-world deployment and sustainability considerations.

\vspace{1cm}

The BYD Yangwang U8 showcases the latest innovations in the EV sector through its advanced battery system, high-performance drivetrain, and environmentally conscious design philosophy. This study explores how the vehicle bridges performance and sustainability, emphasizing BYD’s contribution to clean energy mobility. Additionally, the report highlights real-world insights from site observations and practical evaluations conducted during the visit to Zanvar Industries.

\vspace{1cm}

The analysis concludes that the Yangwang U8 not only sets a benchmark in performance and luxury but also demonstrates the technological direction of future electric mobility. It combines superior efficiency, safety, and environmental responsibility — making it a significant milestone in the evolution of electric vehicles.

\vfill
\begin{center}
    \textit{Keywords:} BYD Yangwang U8, Electric Vehicle, Sustainability, Environmental Impact, Case Study
\end{center}

\newpage
\begin{center}
    \Huge\textbf{Introduction}
\end{center}
\vspace{1cm}

\noindent
The automotive industry is rapidly transitioning toward electric mobility, driven by global concerns about climate change, air pollution, and the depletion of fossil fuels. Electric vehicles (EVs) have emerged as a sustainable alternative to internal combustion engine vehicles, offering significant benefits in terms of energy efficiency, reduced emissions, and lower operating costs.  

\vspace{0.5cm}

Among the latest advancements in this domain is the \textbf{BYD Yangwang U8}, a luxury electric SUV that represents the pinnacle of innovation in the EV industry. Developed by BYD Auto Co. Ltd., this vehicle combines a high-performance drivetrain, sophisticated battery technology, and an intelligent control system designed for both comfort and capability.  

\vspace{0.5cm}

The Yangwang U8 introduces a unique \textbf{quad-motor all-wheel-drive system}, allowing independent torque control for each wheel, enhancing both performance and safety. This design not only improves handling on various terrains but also sets a new benchmark for power distribution in electric mobility.  

\vspace{0.5cm}

This study aims to provide a detailed analysis of the BYD Yangwang U8, including its design, engineering, and environmental performance. It explores how BYD has leveraged technological innovation to produce a vehicle that meets the dual goals of luxury and sustainability. Additionally, the report evaluates the vehicle’s positioning within the broader EV market and examines its contribution toward the future of clean transportation.  

\vfill
\begin{center}
    \textit{Keywords:} BYD Yangwang U8, Electric Mobility, Automotive Innovation, Quad-Motor System, Sustainability
\end{center}
\newpage
\begin{center}
    \Huge\textbf{Description and Specifications}
\end{center}
\vspace{1cm}

\noindent
The \textbf{BYD Yangwang U8} is a luxury off-road electric SUV that combines innovation, performance, and sustainability. Designed on BYD’s advanced \textbf{e4 platform}, the Yangwang U8 represents a breakthrough in electric mobility with its unique quad-motor setup, providing unmatched traction control and driving stability on multiple terrains.  

\vspace{0.5cm}

This section provides a comprehensive overview of the vehicle’s design philosophy, powertrain, and core technical specifications. It highlights how BYD’s engineering excellence translates into both luxury and efficiency while maintaining a strong focus on environmental consciousness.  

\vspace{0.8cm}
\begin{center}
    \Large\textbf{Vehicle Overview}
\end{center}
\vspace{0.4cm}

\noindent
The Yangwang U8’s \textbf{quad-motor all-wheel-drive system} allows each wheel to be controlled independently, resulting in exceptional handling and off-road capability. The e4 platform offers greater safety, precision, and reliability while integrating BYD’s proprietary Blade Battery technology for optimal power management.  

\vspace{0.8cm}
\begin{center}
    \Large\textbf{Technical Specifications}
\end{center}
\vspace{0.4cm}

\begin{itemize}
  \item \textbf{Manufacturer:} BYD Auto Co., Ltd.
  \item \textbf{Model:} Yangwang U8
  \item \textbf{Platform:} BYD e4 Intelligent Platform
  \item \textbf{Powertrain:} Quad Electric Motors (4WD)
  \item \textbf{Total Power Output:} Approximately 1,100 horsepower
  \item \textbf{Battery Capacity:} 49.05 kWh Blade Battery (with hybrid range extender)
  \item \textbf{Range:} Up to 1,000 km (CLTC combined)
  \item \textbf{Acceleration:} 0–100 km/h in 3.6 seconds
  \item \textbf{Charging Type:} DC Fast Charging and AC Slow Charging
  \item \textbf{Seating Capacity:} 7 Passengers
  \item \textbf{Top Speed:} 200 km/h (Electronically Limited)
\end{itemize}

\vspace{0.8cm}

\noindent
The Yangwang U8 blends luxury aesthetics with functional performance, offering both city comfort and off-road dominance. With BYD’s emphasis on battery safety, intelligent control, and sustainable materials, it sets a new benchmark for the global electric SUV market.

\vfill
\begin{center}
    \textit{Keywords:} BYD Yangwang U8, e4 Platform, Quad-Motor System, Blade Battery, Specifications
\end{center}

\newpage
\section{Geotagged Photo(s) of the EVs}
\begin{figure}[h!]
  \centering
  \includegraphics[width=0.9\linewidth]{yangwang.jpg}
  \caption{BYD Yangwang U8 at Zanvar Industries (Geotag: 16.8456° N, 74.6013° E)}
  \label{fig:byd_u8_photo}
\end{figure}

\begin{figure}[h!]
  \centering
  \includegraphics[width=0.9\linewidth]{U8-electric-vehicle_8-jpg.jpg}
  \caption{BYD Yangwang U8 at Zanvar Industries (Geotag: 16.8456° N, 74.6013° E)}
  \label{fig:byd_u8_photo}
\end{figure}

\newpage
\begin{center}
    \Huge\textbf{Environmental Impact Assessment}
\end{center}
\vspace{1cm}

\noindent
The \textbf{BYD Yangwang U8} demonstrates a significant reduction in carbon emissions compared to conventional internal combustion engine (ICE) vehicles. As part of BYD’s commitment to sustainable transportation, the Yangwang U8 incorporates advanced electrification technologies that minimize both direct and indirect environmental impacts throughout its life cycle.  

\vspace{0.5cm}

\noindent
At the heart of its design lies BYD’s proprietary \textbf{Blade Battery technology}, which emphasizes enhanced safety, longer lifespan, and improved recyclability. Unlike traditional lithium-ion batteries, the Blade Battery structure reduces the risk of thermal runaway, resulting in safer operation and lower maintenance costs. Its modular design also facilitates easier recycling and material recovery, reducing long-term waste.  

\vspace{0.5cm}

\noindent
From a manufacturing standpoint, BYD has significantly minimized production emissions by integrating \textbf{renewable energy sources} into its assembly plants. Solar and wind energy are increasingly used to power key manufacturing operations, thereby reducing carbon footprints. Additionally, BYD’s closed-loop battery recycling process ensures valuable materials like lithium, nickel, and cobalt are reclaimed efficiently.  

\vspace{0.5cm}

\noindent
Operationally, the Yangwang U8 contributes to near-zero tailpipe emissions, effectively eliminating pollutants such as CO₂, NOₓ, and particulate matter. Its regenerative braking system further optimizes energy usage, improving overall vehicle efficiency.  

\vspace{0.5cm}

\noindent
However, certain environmental challenges remain. The global supply of rare-earth materials required for electric motors poses sustainability concerns. Additionally, large-scale battery disposal still demands robust recycling infrastructure and government-backed regulations. Continued research and innovation are essential to minimize these issues and ensure a sustainable EV ecosystem.  

\vfill
\begin{center}
    \textit{Keywords:} Environmental Impact, Blade Battery, Carbon Emissions, Renewable Energy, Sustainability
\end{center}

\newpage
\begin{center}
    \Huge\textbf{Analysis and Comparison with Similar EVs}
\end{center}
\vspace{1cm}

\noindent
When compared with other premium electric SUVs such as the \textbf{Tesla Model X}, \textbf{Mercedes EQS SUV}, and \textbf{Rivian R1S}, the \textbf{BYD Yangwang U8} stands out for its distinctive engineering and performance attributes. The Yangwang U8’s \textbf{quad-motor all-wheel-drive system} offers independent torque vectoring, enabling precise control and unmatched off-road performance—an area where many competitors focus primarily on urban or highway driving dynamics.  

\vspace{0.5cm}

\noindent
While the Tesla Model X emphasizes \textbf{aerodynamic efficiency} and long-range capability, and the Mercedes EQS SUV prioritizes interior luxury and refinement, the Yangwang U8 delivers a balance between \textbf{rugged durability and cutting-edge technology}. Its innovative e4 platform, coupled with the Blade Battery system, provides high power output and superior safety, giving it a unique edge in the luxury SUV market.  

\vspace{0.5cm}

\noindent
A significant differentiating factor is the Yangwang U8’s inclusion of a \textbf{range extender}, which allows longer travel distances compared to pure electric competitors. This hybrid capability enhances practicality, especially in regions with limited charging infrastructure. Moreover, BYD’s manufacturing efficiency and cost optimization strategies make the U8 more competitively priced without compromising quality.  
\newpage
\vspace{0.8cm}
\begin{center}
    \Large\textbf{Comparative Overview}
\end{center}
\vspace{0.4cm}

\begin{table}[h!]
\centering
\renewcommand{\\arraystretch}{1.3}
\begin{tabular}{|l|c|c|c|}
\hline
\textbf{Parameter} & \textbf{BYD Yangwang U8} & \textbf{Tesla Model X} & \textbf{Rivian R1S} \\\hline
Power Output (hp) & 1,100 & 1,020 & 835 \\\hline
Range (km) & 1,000 (CLTC) & 580 (EPA) & 500 (EPA) \\\hline
0–100 km/h (s) & 3.6 & 2.6 & 3.0 \\\hline
Battery Capacity (kWh) & 49.05 + Extender & 100 & 135 \\\hline
Drive Type & Quad Motor 4WD & Dual/Triple Motor AWD & Quad Motor AWD \\\hline
Base Price (Approx.) & \$150,000 & \$110,000 & \$95,000 \\\hline
\end{tabular}
\end{table}

\vspace{0.6cm}

\noindent
From this comparison, it is evident that the Yangwang U8 positions itself as a \textbf{performance-oriented and feature-rich SUV} capable of handling both luxury urban commutes and rugged off-road expeditions. BYD’s innovation in energy management and motor control technology provides it with a strategic advantage in the evolving global EV market.

\vfill
\begin{center}
    \textit{Keywords:} BYD Yangwang U8, Tesla Model X, Rivian R1S, Performance Comparison, Luxury Electric SUV
\end{center}

\begin{table}[h!]
\centering
\caption{Comparison of BYD Yangwang U8 with Similar EVs}
\begin{tabular}{lccc}
\hline
\textbf{Parameter} & \textbf{BYD Yangwang U8} & \textbf{Tesla Model X} & \textbf{Rivian R1S} \\
\hline
Power Output (hp) & 1,100 & 1,020 & 835 \\
Range (km) & 1,000 (CLTC) & 580 (EPA) & 500 (EPA) \\
0–100 km/h (s) & 3.6 & 2.6 & 3.0 \\
Battery Capacity (kWh) & 49.05 + extender & 100 & 135 \\
Drive Type & Quad Motor 4WD & Dual/Triple Motor AWD & Quad Motor AWD \\
\hline
\end{tabular}
\end{table}

\newpage
\begin{center}
    \Huge\textbf{Conclusion}
\end{center}
\vspace{1cm}

\noindent
The \textbf{BYD Yangwang U8} exemplifies the future of electric mobility by seamlessly combining extreme performance, innovative engineering, and environmental responsibility. It represents a milestone in the evolution of electric vehicles, showing how cutting-edge technology can coexist with sustainability and practicality.  

\vspace{0.5cm}

\noindent
Through its \textbf{quad-motor drivetrain}, advanced \textbf{Blade Battery system}, and \textbf{range-extending technology}, the Yangwang U8 redefines what a luxury electric SUV can achieve. The vehicle’s emphasis on safety, performance, and efficiency sets new benchmarks not only for BYD but also for the broader EV industry. Its ability to adapt to diverse driving conditions—ranging from urban commuting to off-road terrains—further enhances its market appeal.  

\vspace{0.5cm}

\noindent
Environmentally, the Yangwang U8 demonstrates how electric vehicles can drastically reduce emissions and dependence on fossil fuels. With BYD’s efforts to integrate renewable energy sources into its production chain and implement closed-loop battery recycling, the company has shown strong commitment toward sustainable development.  

\vspace{0.5cm}

\noindent
However, to maximize long-term sustainability, continuous improvements in renewable energy sourcing, rare-earth material alternatives, and recycling infrastructure are essential. Such efforts will ensure that the environmental benefits of electric mobility are fully realized across the entire lifecycle of the vehicle.  

\vspace{0.5cm}

\noindent
In summary, the BYD Yangwang U8 stands as a \textbf{technological and environmental benchmark}—a symbol of how innovation and sustainability can move in harmony toward a cleaner, smarter automotive future.

\vfill
\begin{center}
    \textit{Keywords:} BYD Yangwang U8, Electric Mobility, Sustainability, Blade Battery, Innovation
\end{center}



\end{document}
